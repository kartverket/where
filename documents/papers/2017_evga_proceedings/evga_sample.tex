\documentclass[natbib,twocolumn,twoside]{svmultiag}
%
%\usepackage[dvips]{graphicx}
\usepackage{graphicx}
% ============================================================================
\graphicspath{{./figures/}}

\title*{CONT08 - First Results and High Frequency Earth Rotation}
\titlerunning{CONT08 - First Results and HF-ERPs}

\author{T. Artz, S. B{\"o}ckmann, A. Nothnagel}
\authorrunning{Artz et al.}

\institute{Thomas Artz, Axel Nothnagel and Sarah Tesmer %
           \at Rheinische Friedrich-Wilhelms Universit{\"a}t Bonn, IGG,
               Nu{\ss}allee 17, D-53115 Bonn, Germany
           % Additional author \at Affiliation, Adress, ...
          }
% ============================================================================
\begin{document}
\maketitle
% ----------------------------------------------------------------------------
\abstract{
First results from the most recent continuous VLBI campaign (CONT08) are
shown. 
CONT08 took place in August 2008. One of the scientific goals was the
generation of high precision Earth Orientation Parameters (EOP) with sub-daily
resolution.
A general quality assessment of CONT08 is performed investigating station
position variations and daily EOP. 
In addition, high-resolution Earth rotation time series are generated in a
way that ensures consistency over the whole time span. 
Here, we demonstrate the effect of a modified scheduling procedure concerning
\mbox{(in-)consistency} of sub-daily EOP at session borders.
Results of prior continuous campaigns have differed significantly from each 
other in amplitude and phase of the spectral components.
Thus, we also compare the CONT08 amplitude spectra with results from CONT02 
and CONT05.
}

\keywords{CONT08, High Frequency Earth Rotation}


% ----------------------------------------------------------------------------
\section{Introduction}                                \label{sec:introduction}
% ----------------------------------------------------------------------------
The International VLBI Service for Geodesy and Astrometry (IVS) conducts 
continuous VLBI observations of up to 14 days duration (CONT) in irregular
intervals. 
In contrast to the routine 24~h VLBI observations, which are performed two or
three times per week, those CONT-campaigns are around-the-clock observations
over several days.
The last three CONT campaigns carried out in 2002, 2005 and 2008 took place
over fortnightly time-spans.
CONT-sessions  should always demonstrate the highest available accuracy that
can be achieved by VLBI observations.
Additionally, one of the main scientific goals is the generation and
interpretation of highly resolved Earth orientation parameters (EOP) due to
the continuity of the campaign.

Most recently, CONT08 took place in 2008 from August 12-26 with individual
sessions being set up for 0~h to 24~h UTC each.
The observing network was arranged by 11 globally distributed VLBI telescopes
(s. Fig. \ref{fig:network}).
The network has a strong European part since the stations in North-America at
Fairbanks and Algonquin Park (that were part of prior continuous campaigns)
were powered off and replaced by Medicina and Zelenchukskaya.
% ............................................................................
% Fig.: CONT08 network
\begin{figure}[tb]
         \includegraphics[width=7.4cm]{artz01}
         \caption{CONT08 observing network}
         \label{fig:network}
\end{figure}
% ............................................................................

There are several changes in the scheduling of CONT08 with respect to prior
continuous campaigns.
The keypoint is the necessity of system checks that have to be performed on a
daily basis.
For CONT08 these have been decoupled in view on the observing sites.
Prior to CONT08, the system checks were performed in the last $\approx$ 30~min
of each session for all stations.
Now, subsequent 2~h slots for each site have been planned for each session
with exception of the first one.
In those slots, the 30~min system checks should be performed (see Tab. 
\ref{tab:slots}).
In reality, this idea has been fullfilled.
Moreover, some stations performed system checks so quickly that nearly no
gaps are visible in the observations.
% ............................................................................
% table with time slots
\begin{table}[t]
 \centering
 \begin{tabular}{c||c|c}
    Station & Weekends [UT] & Weekdays [UT] \\ 
    \hline
    Hr  & 13-15& 12-14\\
    Kk  & 00-02& 18-20\\
    Mc  & 15-17& 06-08\\
    Ny  & 17-19& 10-12\\
    On  & 11-13& 08-10\\
    Sv  & 05-07& 04-06\\
    Tc  & 21-23& 16-18\\
    Ts  & 07-09& 00-02\\
    Wf  & 19-21& 14-16\\
    Wz  & 07-09& 18-20\\
    Zc  & 09-11& 02-04
 \end{tabular}
 \caption{Time slots for system checks during CONT08
          (http://ivscc.gsfc.nasa.gov/program/cont08/cont08-notes.txt)}
 \label{tab:slots}
\end{table}
% ............................................................................

Focussing on the observations itself, one can state that each session has an
overall number of 9000 to 11000 observations.
This is a great increase w.r.t. CONT02 (mean: 3000) and CONT05 (mean: 6000)
due to a raise of the recording rate and, thus, shorter scan lengths.
The variation between the individual sessions mainly depends on station
problems.
TIGO/Conception missed 3 sessions and had several interruptions of up to 6~h.
Zelenchukskaya is absent in one session because of lost disks. 
Furthermore, post fit residuals of observations with Zelenchukskaya show 
bi-modal patterns suggesting the existence of sub-ambiguities. 
Since these could not be fixed, the observations affected have been
eliminated from the solution.

In this paper, first results of the analysis of CONT08 observations are given
with a special focus on the estimation of sub-daily EOP.
In particular, the impact of the modified scheduling on the estimates is
evaluated.
Furthermore, CONT-campaigns of 2002, 2005 and 2008 are compared on the basis
of station position estimates and the frequency representation of polar
motion.


% ----------------------------------------------------------------------------
\section{Solution Description}                                 \label{sec:sol}
% ----------------------------------------------------------------------------
Three different solutions have been performed for the analysis of CONT08 
as well as of CONT02 and of CONT05.
\begin{enumerate}
\item Solution for session-wise station position estimates. The datum 
      defect has been solved by NNR/NNT conditions w.r.t. ITRF2005
      over the whole set of stations.
\item EOP solution with daily estimates of polar motion, $\Delta$UT1
      and their rates as well as nutation offsets.
\item EOP solution with hourly PM and $\Delta$UT1 parameterized as
      continuous-pice-wise-linear-functions (CPWLF). Nutation
      parameters are fixed to a~priori values estimated in a separate
      solution.
\end{enumerate}
In solutions 2. and 3., the station positions are estimated only once for the
middle epoch of the fortnightly time-span to de-correlate the estimates as
shown by \cite{Artz2007}.

All of these solutions have in common the same modelling and parameterization
of nuisance parameters (Clocks and atmospheres).
Source positions are fixed to ICRF and its extensions.
Station clocks are estimated w.r.t. Kokee Park clock by a $2^{nd}$ order
polynomial with additional clock parameters modelled by 60~min CPWLF. 
Troposphere parameters are estimated as CPWLF, too.
The zenith wet delay is parameterized with a temporal resolution of 20~min 
and gradients in 12~h intervals.

For all of these solutions a~priori EOP are taken from USNO finals 
\footnote{ftp://maia.usno.navy.mil/ser7/finals.daily}.
Nutation is modelled by IAU2000A \citep{iers_conv2004} plus additional 
corrections from a global VLBI solution.
Ocean loading is modelled according to FES2004 \citep{fes2004}, furthermore,
thermal expansion of the radio telescopes \citep{nothnagel2008} and
atmospheric pressure loading \citep{aplo} have been applied.

The solutions presented here are all run in a two step procedure.
First, the data is processed by the VLBI analysis software Calc/Solve 
\citep{solve} and the normal equation system (NEQ) is exported.
In a second step the NEQ is solved with a Perl backend to Calc/Solve.
To ensure the continuity of the campaign and to stabilize parameters at the 
session borders, a modified solution strategy is used.
The NEQ are build up for each single session.
Clock parameters are treated as session parameter, thus, they have been
reduced from each individual session NEQ.
Finally, the individual NEQ are added to a single one for the complete
campaign by adding elements of parameters of the same type and the same epoch.
Thereby, parameters at the session borders are stabilized and estimated only
once \citep{Artz2007}.


% ---------------------------------------------------------------------------
\section{Quality Assessment}
% ---------------------------------------------------------------------------
On the basis of solutions 1 and 2, an assessment of the quality of CONT08
can be achieved.
On the one hand, station position estimates will be compared within the
campaign.
On the other hand, the repeatabilities are compared to those in CONT02
and CONT05.
Furthermore, an external validation is performed on the basis of daily EOP.
These are compared to those estimated from GPS observations.

% ............................................................................
% Fig: Station position estimates
\begin{figure}[tb]
         \centering
         \includegraphics[width=.5\textwidth]{artz04}
         \includegraphics[width=.5\textwidth]{artz03}
         \includegraphics[width=.5\textwidth]{artz02}
         \includegraphics[width=.3\textwidth]{artz05}
         \caption{Differences of the individual session estimates to the 
                  campaign mean in CONT08.}
         \label{fig:sta_est}
\end{figure}
% ............................................................................
The differences of the individual station position estimates and the 
campaign mean are quite homogeneous (Fig.~\ref{fig:sta_est}).
Most of them are below 10~mm for the horizontal and 20~mm for the up 
component.
Only the variability of the east component of Kokee Park and Tsukuba is large
compared to the other sites.
Nevertheless, the impact on the estimation of other parameters is not big 
e.g. excluding these sites from the datum definition has nearly no impact 
on the other station positions.
The bigger deviations of TIGO/Conception in the session of August 18$^{th}$
can be explained by the minor contribution to this session.
Here, TIGO had only a few hours of observations due to some station problems
and, thus, the error bars of these estimates are huge.
As a consequence, there is no big impact on the estimation of other parameters
as well.

In general, the station position estimates are as stable as those of the prior
campaigns.
Figure~\ref{fig:sta_rep} shows the station position repeatabilities for
CONT02, CONT05 and CONT08.
Only observing sites that were in at least two campaigns are displayed.
% ............................................................................
% Fig: Station posistion RMS differences
\begin{figure}[t]
 \centering
 \includegraphics[width=.5\textwidth, trim=0 .06cm 0 0, clip]{artz08}
 \includegraphics[width=.5\textwidth, trim=0 .06cm 0 0, clip]{artz07}
 \includegraphics[width=.5\textwidth]{artz06}
 \caption{Station position repeatabilities for each campaign. 
          The datum definition has been applied by NNR/NNT conditions
          over all stations in CONT02 (red) and over all stations
          but Kokee Park and Tsukuba in CONT05 (black) and CONT08 
          (purple). The two stations that were active in CONT08, only,
          (Medicina and Zelenchukskaya) are excluded from this figure.}
 \label{fig:sta_rep}
\end{figure}
% ............................................................................
Most of the repeatabilities in CONT08 are of comparable size as in the prior
campaigns or even better.
Only the repeatabilities of Kokee Park, Tsukuba and Westford have become
worse.
The main cause can be seen in the switch-off of the telescopes at Fairbanks 
and Algonquin Park reducing the geometric links between these sites.
As a general consequence, it can be stated that the CONT08 network was not
that well distributed over the Earth as it was in CONT05.

Calculating the WRMS of the daily EOP differences between the VLBI estimates
and external EOP series leads to an insight in the quality of the EOP from a 
continuous campaign.
These WRMS differences are shown in Fig~\ref{fig:erp_rep}, where the official
combined EOP series of the International GNSS Service
\footnote{ftp://igscb.jpl.nasa.gov/pub/product/igs00p03.erp.Z} has been used
for polar motion.
For $\Delta$UT1 the USNO finals have been used to calculate the EOP
repeatabilities.
The differences have been detrended to eliminate the impact of differences
in the underlying terrestrial reference frame.
The UT1 variations from the three continuous VLBI campaigns agree with the 
reference series at the same level, whereas the CONT08 polar motion agrees
much better with the IGS time series than those from CONT02 or CONT05.
% ............................................................................
% Fig: ERP RMS differences
\begin{figure}[t]
 \centering
 \includegraphics[width=.3\textwidth]{artz09}
 \caption{EOP repeatabilities for each campaign. 
          The WRMS values are calculated w.r.t. IGS EOP series
          for polar motion and USNO finals for $\Delta$UT1.}
 \label{fig:erp_rep}
\end{figure}
% ............................................................................

% ---------------------------------------------------------------------------
\section{Sub-daily Earth Rotation}
The estimation of EOP with a sub-daily resolution from CONT campaigns is of
great interest.
Hourly EOP estimates over a time-span of two weeks provide the opportunity
to analyze the characteristics of the EOP in the frequency domain.

Furthermore, the time series of hourly EOP shows the effects of the modified
scheduling used in CONT08.
Figure \ref{fig:xpole} displays the X~pole differences w.r.t. the IGS time
series for CONT05 and CONT08 from a solution where each session is analyzed
independently.
There are huge outliers in the CONT05 time series at the session borders.
Those are due to the lack of observations in the last $\approx$~30~min of
each 24~h block where the system checks were performed.
Such outliers are not visible in the CONT08 time series.
For CONT08 the scheduling was changed in a way that the observations are
really continuous as described in Sec.~\ref{sec:introduction}.
Thus, the estimates in the last interval of each individual session are
already stabilized through this type of scheduling.
% ............................................................................
% Fig: hourly resolved X-pole
\begin{figure}[t]
 \centering
 \includegraphics[width=.4\textwidth]{artz10}
 \includegraphics[width=.4\textwidth]{artz11}
 \caption{X pole differences to IGS with a hourly resolution from CONT05 
          (upper plot) and CONT08 (lower plot) from an independent solution.}
 \label{fig:xpole}
\end{figure}
% ............................................................................

In addition to the scheduling, the modified solution strategy of stacking the
normal equations helps to further improve the sub-daily EOP estimates.
Here, parameters at the session boundaries are estimated using observations
from the last interval of the first session and the first interval of the
second session and so on.
The resulting time series are shown in Fig.~\ref{fig:stack}.
Nearly all of the outliers in CONT05 are eliminated or at least minimized
by the modified stacking approach.
A minor improvement can be seen for CONT08 as well.
% ............................................................................
% Fig: hourly resolved X-pole (modified solution approach)
\begin{figure}[t]
 \centering
 \includegraphics[width=.4\textwidth]{artz12}
 \includegraphics[width=.4\textwidth]{artz13}
 \caption{X pole differences to IGS with a hourly resolution from CONT05 
          (upper plot) and CONT08 (lower plot) where the modified
          solution strategy has been applied.}
 \label{fig:stack}
\end{figure}
% ............................................................................

In the spectral domain several authors have reported about a retrograde
ter-diurnal signal in the CONT02 Polar Motion data (e.g. \cite{haas2006a}).
This signal is visible in our analysis, too, as it is shownin
Fig.~\ref{fig:spectrum}.
However, there is no such signal present in the other campaigns.
To derive the spectra, total EOP series are reduced by the IERS sub-daily 
tidal model \citep{iers_conv2004}.
The reason for this phenomena is still an open issue and under investigation.
Subsequently, those values have been detrended by CPWLF with a resolution of 
one day.
Afterwards, a FFT of the residuals has been calculated.
Some peaks can be seen at the diurnal and semi-diurnal bands.
The prograde diurnal and semi-diurnal as well as the semi-diurnal retrograde
signal can be interpreted as inconsistency between the IERS sub-daily tidal
model and the VLBI observations.
The remaining retrograde diurnal signal is due to mismodelling of nutation.
Nutation is fixed to a~priori values for the estimation of sub-daily
EOP, thus, some of the remaining signal appears as a near diurnal retrograde 
term.
% ............................................................................
% Fig: amplitude spectra
\begin{figure}[t]
 \centering
 \includegraphics[width=.5\textwidth]{artz14}\\
 \includegraphics[width=.5\textwidth]{artz15}\\
 \includegraphics[width=.5\textwidth]{artz16}
 \caption{Polar Motion spectra from CONT-campaigns.}
 \label{fig:spectrum}
\end{figure}
% ............................................................................
In the $\Delta$UT spectra no significant variations besides the diurnal and
semi-diurnal bands can be seen.
There is a small irregular variation at a period of 6h in the CONT05 data
only.

% ---------------------------------------------------------------------------
\section{Conclusions}
% ---------------------------------------------------------------------------
CONT08 serves as a high quality continuous data-set from VLBI observations.
The variations of the station position estimates is quite homogeneous 
besides some bigger variations in the east component of Kokee Park and Tsukuba.

The modified scheduling improves the estimation of parameters with a sub-daily 
resolution as shown for the hourly EOP estimates.
Not only EOP but all parameters with a sub-daily resolution as zenith wet 
delay or clock parameters benefit from this procedure.

The sub-daily EOP estimates match the GPS series quite well for all campaigns.
The derived spectra for polar motion are inhomogeneous.
The peaks at the well known tidal bands with periods around 12h and 24h vary.
Furthermore, the retrograde ter-diurnal term is present in CONT02 polar motion
but not in CONT05 or CONT08.
For variations in $\Delta$UT, no significant irregular variations could be 
found.

% ---------------------------------------------------------------------------
\begin{thebibliography}{7}
\providecommand{\natexlab}[1]{#1}
\providecommand{\url}[1]{\texttt{#1}}
\expandafter\ifx\csname urlstyle\endcsname\relax
  \providecommand{\doi}[1]{doi: #1}\else
  \providecommand{\doi}{doi: \begingroup \urlstyle{rm}\Url}\fi

\bibitem[Artz et~al.(2007)Artz, B{\"o}ckmann, Nothnagel, and Tesmer]{Artz2007}
T.~Artz, S.~B{\"o}ckmann, A.~Nothnagel, and V.~Tesmer.
\newblock ERP time series with daily and sub-daily resolution determined from
  CONT05.
\newblock In J.~B{\"o}hm, A.~Pany, and H.~Schuh, editors, \emph{Proceedings of
  the 18th Workshop Meeting on European VLBI for Geodesy and Astrometry},
  volume~79 of \emph{Geowissenschaftliche Mitteilungen, Schriftenreihe
  Vermessung und Geoinformation der TU Wien}, pages 69--74. TU Wien, 2007.

\bibitem[Haas and W{\"u}nsch(2006)]{haas2006a}
R.~Haas and J.~W{\"u}nsch.
\newblock Sub-diurnal earth rotation variations from the VLBI CONT02 campaign.
\newblock \emph{J. Geodyn.}, 41:\penalty0 94--99, 2006.

\bibitem[{{Letellier} et~al.(2004){Letellier}, {Lyard}, and {Lefevre}}]{fes2004}
T. {Letellier}, F.~{Lyard}, and F.~{Lefevre} (2004), {The new global tidal solution: FES2004}, 
\textit{Proceeding of the Ocean Surface Topography Science Team Meeting, St. Petersburg, Florida} 

\bibitem[{McCarthy} and {Petit}(2004)]{iers_conv2004}
D.~D. {McCarthy} and G.~{Petit}.
\newblock \emph{IERS Conventions (2003)}.
\newblock IERS Conventions (2003).~Dennis D.~McCarthy and G{\'e}rard Petit
  (eds.), International Earth Rotation and Reference Systems Service
  (IERS).~IERS Technical Note, No.~32, Frankfurt am Main, Germany: Verlag des
  Bundesamtes f{\"u}r Kartographie und Geod{\"a}sie, ISBN 3-89888-884-3, 2004,
  127 pp.,  2004.

\bibitem[Nothnagel(2008)]{nothnagel2008}
A.~Nothnagel.
\newblock Conventions on thermal expansion modelling of radio telescopes for
  geodetic and astrometric VLBI.
\newblock \emph{J. Geodesy}, 2008.
\newblock \doi{10.1007/s00190-008-0284-z}.

\bibitem[Petrov(2008)]{solve}
L.~Petrov.
\newblock Mark-5 vlbi analysis software calc/solve.
\newblock Web document http://gemini.gsfc.nasa.gov/solve/, 07 2008.

\bibitem[{Petrov} and {Boy}(2004)]{aplo}
L.~{Petrov} and J.-P. {Boy}.
\newblock Study of the atmospheric pressure loading signal in very long
  baseline interferometry observations.
\newblock \emph{Journal of Geophysical Research (Solid Earth)}, 109:\penalty0
  3405--+, Mar. 2004.
\newblock \doi{10.1029/2003JB002500}.

\end{thebibliography}
\end{document}
