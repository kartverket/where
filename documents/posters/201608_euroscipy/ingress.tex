A combination of data from quasars (VLBI), lasers (SLR) and satellites (GPS and
DORIS) is used to develop a coordinate system for the Earth, which is necessary
to monitor the changes of the Earth's surface: Land rising after the last ice
age, sea level increases and the tectonic plates moving several centimeters
every year. Analysis from several research institutes around the world is
combined in order to establish this coordinate system, known as the
International Terrestrial Reference Frame (ITRF). Traditionally, most of this
analysis has been done using Fortran code. At Kartverket (the Norwegian Mapping
Authority) we are currently developing Where, a new software written in Python
that will take part in this international collaboration.

\endinput
