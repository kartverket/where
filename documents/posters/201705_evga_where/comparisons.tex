In 2015 and 2016 Grzegorz Klopotek carried out a VLBI Analysis Software
Comparison Campaign (VASCC). The goal of the campaign was to compare computed
theoretical delays from different software packages. In total 11 different
software packages contributed to the campaign and the results where presented at
the IVS General Meeting in South Africa in 2016~\cite{klopotek2016}.

The Norwegian Mapping Authority provided solutions to VASCC using the legacy
software GEOSAT~\cite{kierulf2010}. However, as development of the new software
progressed we redid the analysis from the VASCC campaign using Where and
compared the results with the computed theoretical delays from the
c5++~\cite{hobiger2010} and VieVS~\cite{boehm2012} softwares.

The VASCC dataset includes two networks of stations: one with four stations on
the southern hemisphere (SH) and one with five stations on the northern
hemisphere (NH).  Virtual observations of one radio source for each network are
scheduled every minute for 15 days, from June 22nd to July 7th 2015. A leap
second is introduced at midnight June 30th.

The VASCC delay model includes geometric and gravitational delay from the IERS
2010 Conventions~\cite{iers2010}. Also the delay through the
troposphere~\cite{iers2010} (hydrostatic delay with GMF mapping function) and
delay due to thermal deformations (with constant temperatures) and axis offset
as described by~\cite{nothnagel2009} are included. Delays due to cable
calibration, the ionosphere and clocks are ignored. Site displacement models
includes solid earth tides, solid earth pole tides, ocean tidal loading
(FES2004) and ocean loading pole tides according to~\cite{iers2010}. The version
of the IERS Conventional mean pole model used in VASCC is 2010. Atmospheric
pressure loading and eccentricity vectors are ignored. The apriori EOP time
series is corrected for ocean tides and liberation effects with periods less
than two days according to~\cite{iers2010}.

In the original campaign six software packages got sub-millimeter agreement when
comparing the RMS of the difference in theoretical delays. The absolute value of
the largest difference in residual was 2.68~mm and the smallest difference was
0.83~mm. The solution from two of the these six software packages,
c5++~\cite{hobiger2010} and VieVS~\cite{boehm2012}, were compared with Where and
the results are summarized in tables~\ref{tbl:vascc_rms}
and~\ref{tbl:vascc_max}. Figures~\ref{fig:where_c5++_sh}, \ref{fig:where_vievs_sh}
and~\ref{fig:c5++_vievs_sh} show the difference between Where, c5++ and VieVS
for the southern network.

These results indicate that the VLBI delay model in Where is consistent with
existing software packages and current conventions. The VASCC dataset
has been valuable in the development and testing of Where. An extension of the
campaign to include for instance the VMF1 mapping function and partial
derivatives would also be useful.

\endinput
