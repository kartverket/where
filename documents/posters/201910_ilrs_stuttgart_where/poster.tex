\documentclass{beamer}
\usepackage{lipsum}

\mode<presentation>{\usetheme{Ringerike}}
\usepackage[english]{babel}
\usepackage[latin1]{inputenc}
\usepackage{multicol}
\usepackage{tabularx}

%\usepackage{amsmath,amsthm, amssymb, latexsym}
%\usepackage{lipsum}
%\boldmath
\usepackage[orientation=portrait, size=a0]{beamerposter}

\addtobeamertemplate{block begin}{}{\setlength{\parskip}{30pt plus 1pt minus 1pt}}
\setbeamertemplate{caption}[numbered]

% Bibliography colors and labels
\setbeamertemplate{bibliography item}{\color{white}\ \insertbiblabel}
\setbeamercolor{bibliography entry item}{fg=white}
\setbeamercolor{bibliography entry author}{fg=white}
\setbeamercolor{bibliography entry title}{fg=white} 
\setbeamercolor{bibliography entry location}{fg=white} 
\setbeamercolor{bibliography entry note}{fg=white}

\title{Where}
\subtitle{Software for Geodetic Analysis}
\author{Ingrid Fausk, Michael D\"ahnn, Ann-Silje Kirkvik}
\newcommand{\contact}{ingrid.fausk@kartverket.no\\ {\url https://github.com/kartverket/where}}
\institute{Norwegian Mapping Authority, Geodetic Institute}
\date{October 2019}

\usebackgroundtemplate{\includegraphics[width=\paperwidth]{figure/earth}}

\begin{document}
\begin{frame}[t]

  % Top title area
  %_____________________________________________________________________________________________
  \color{white}
  \vspace*{4cm}
  \begin{columns}
    \begin{column}[t]{.97\textwidth}
      {\bfseries\fontsize{88}{120}\selectfont \inserttitle}
      {\fontsize{88}{120}\selectfont\kern2cm---\kern2cm\insertsubtitle}
    \end{column}
  \end{columns}

  \vspace*{2cm}
  \begin{columns}
    \begin{column}[t]{.24\textwidth}
      {\fontsize{30}{36}\selectfont\insertauthor\\[0.5cm]
        \fontsize{30}{36}\selectfont{\itshape\insertinstitute}\\ \ \\
        \fontsize{24}{18}\selectfont\texttt{\contact}}\\
        \vspace*{7cm}
        {\color{white}\tiny PHOTO: GETTY IMAGES}
    \end{column}

    \begin{column}[t]{.70\textwidth}
      {\fontsize{30}{36}\selectfont\setlength{\parskip}{15pt}At the end of 2016 the European Commission has declared the Galileo Initial Services. The declaration means that the Galileo satellites and ground infrastructure are operational and ready for positioning, navigation and timing to users on the way to full operational capability in 2020.

Galileo performance monitoring plays an important role for testing and verifying the initial services and to ensure the provision of high quality satellite data to users. One of the key performance indicators is the signal-in-space range error (SISRE). SISRE represents the error budget related to the control and space segment of Global Navigation Satellite Systems and can be determined by comparing broadcast against precise ephemerides.

The SISRE analysis is implemented in the geodetic analysis software \textbf{Where}. We will describe the used methodology and we will present the first results of the Galileo orbit performance monitoring based on the SISRE analysis with \textbf{Where}.

\vspace{2cm}
\endinput
}
    \end{column}

  \end{columns}
  
  \vspace*{3cm}

  \begin{columns}
    % Content area
    \begin{column}[t]{.9\textwidth}
      \begin{block}{Where}
        \begin{multicols}{2}
          \documentclass[12pt, english]{beamer}

% Packages
\usepackage{xcolor}
\usepackage{eulervm}
\usepackage[utf8]{inputenc}

% Theme
\mode<presentation>
{
  \usetheme{Honefoss}
  \setbeamertemplate{blocks}[rounded]
}

\newcommand{\comment}[1]{{\slshape\color{kvred}#1}}

\title{Where}
\subtitle{A Geodetic Software}
\author{Ingrid Fausk, Michael Dähnn, Ann-Silje Kirkvik}
\date{April 6, 2019}

\begin{document}
\frame[plain]{\titlepage}

\begin{frame}{Where Timeline}
  \begin{itemize}
    \item 2015: Start
    \item 2018: First release as an open source project on GitHub
    \item 2019: Two proposed IVS analysis centers with Where: 
       \begin{itemize}
         \item Kartverket, Norway
         \item Instituto Geográfico Nacional, Spain
       \end{itemize}
    \item 2020: IVS Analysis centers with Where?
    \item 2022: ILRS Analysis center with Where?
  \end{itemize}
\end{frame}

\begin{frame}{Live Demo of Where}
  \begin{itemize}
    \item Running \emph{Where}
    \item Running \emph{There}, a companion tool for visualizing results
    \item Status, discussion
    \item Bugs...
  \end{itemize}
  \begin{figure}
    \begin{flushleft}
      \includegraphics[width=2.5cm]{bug.jpg}
    \end{flushleft}
  \end{figure}     
\end{frame}

\begin{frame}{The Technical Stuff}

The \textbf{Where} software is mainly being written in \emph{Python}

  \begin{itemize}
    \item Cross-platform: Runs on Linux, Mac, Windows
    \item Solid, flexible and fast libraries like \texttt{numpy}, \texttt{astropy}, \texttt{matplotlib} and \texttt{scipy} are available
    \item We use a \textbf{HDF5}-based format for internal data storage
    \item \emph{Python} has effective interfaces to \emph{C} and \emph{Fortran} code, and we use the \textbf{SOFA} and \textbf{IERS} software libraries directly
    \item Orbit integrator using \emph{Cowell} method written in \emph{Python}. 
  \end{itemize}
\end{frame}

\end{document}

        \end{multicols}
      \end{block}
    \end{column}
  \end{columns}
  
  \vspace*{3cm}
 
  \begin{columns}
    \begin{column}[t]{.37\textwidth}

      \begin{figure}
        \includegraphics[width=\textwidth]{figure/there}
        \caption{A screenshot of There - a graphical tool developed
	  to look into the results and analysis done by Where. The plot shows
          the observation residuals for the four satellites Lageos1,
          Lageos2, Etalon1 and Etalon2, in meters, for a one week run.} \label{fig:there}
      \end{figure}
    \end{column}
     
    \begin{column}[t]{0.4\textwidth}
      \begin{block}{Future Analysis Centers for VLBI and SLR with Where}
    %    \begin{multicols}{2}
          
The Norwegian Mapping Authority (NMA) is an associated analysis center within
the IVS and ILRS. Both the NMA and the Instituto Geogr\'{a}fico Nacional,
Spain, are in a test phase of deliveries of VLBI analysis results to the IVS
with the Where software. Some of our activities in VLBI is documented
in~\cite{kirkvik2017b} and~\cite{kirkvik2018}. 

Our goal is to be able to contribute to the ILRS after some improvements of the
software, and in the future recieve full status as operational analysis center
for both VLBI and SLR. 

Sharing and cooperating with other institutions is made possible by making
Where an open source project on GitHub.

\endinput

    %    \end{multicols}
      \end{block}
    \end{column}
  \end{columns}

  \vspace*{3cm}
  
  \begin{columns} 
    \begin{column}[t]{.48\textwidth}
      \begin{table}
        \color{black}
\begin{tabularx}{\columnwidth}{l|X}
  Orbit models & Earth gravity field EGM2008 (with Solid Earth and Ocean Tides),
  Gravity field of Sun, Moon and planets, Solar radiation pressure, Relativistic effects, Empirical accelerations: Constant and once-per-rev along track and cross track \\
  \hline
  Ephemerides & DE405, DE421, DE430 \\
  \hline
  Displacement models & Atmospheric loading, Eccentricity vector, Ocean loading,
  Ocean pole tides, Solid Earth tides, Solid Earth pole tides \\
  \hline
  Delay models & Mendes-Pavlis troposphere models, Center of mass corrections from ILRS Data Handling file, Relativistic corrections\\
  \hline
  Estimation & Continuous piecewise linear Kalman Filter: EOP, Station positions, Range- and Time bias. 
\end{tabularx}

\endinput

        \caption{Models and apriori data supported by Where}
        \label{tbl:models}
      \end{table}
    \end{column}

    \begin{column}[t]{.26\textwidth}
      \begin{block}{References}
%        \begin{multicols}{2}
        \footnotesize
\vspace*{-1cm}
\bibliographystyle{../../where}
\bibliography{../../where}

\endinput

%        \end{multicols}
      \end{block}
    \end{column}
  \end{columns}

\end{frame}
\end{document}
