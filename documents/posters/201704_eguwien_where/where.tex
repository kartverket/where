Where is a new software for geodetic analysis, currently being developed at the
Norwegian Mapping Authority (Kartverket). Where is built on our experiences with
the GEOSAT software~\cite{kierulf2010}, and will be able to analyse and combine
data from VLBI, SLR, GNSS and DORIS.


{\large\bfseries Why?}\\

The last decade the Norwegian Mapping Authority has increased its contributions
on global reference frames.  We are currently building a new fundamental station
at Ny-{\AA}lesund supporting VLBI, SLR, GNSS and DORIS. We have taken initiative
to passing a UN resolution on global geodesy and the importance of Global
Geodetic Reference Frames~\cite{un_ggrf}. The Where project is the third leg in
this effort, developing a new tool for the analysis of geodetic data and giving
us more insight into such analysis.


{\large\bfseries How?}\\

Where is mainly implemented in Python. The Python ecosystem for data science is
very rich and Where utilizes several well known packages such
as \texttt{numpy}, \texttt{scipy} and \texttt{matplotlib}, as well as more
specialized packages like \texttt{astropy}~\cite{astropy2013}
and \texttt{jplephem}~\cite{jplephem}. In addition, Python easily interfaces
with other languages like C and Fortran, and Where uses
the \texttt{SOFA}~\cite{sofa_software} and \texttt{IERS}~\cite{iers_software}
Fortran libraries directly.

Using Python has several advantages, including rapid development, flexible and
easy to read code. It has some challenges, in particular pure Python code can be
slow. However, using \texttt{numpy}, doing most operations on vectors (instead
of sequentially) and caching results when possible gives us a fairly fast
program. There are now many tools for optimization of Python, and there is always
the option to rewrite critical components in lower level languages.

Where is a command line tool, and comes with a configuration file that can be
used to customize the behavior of the program. This makes the program quite easy
to use, while at the same time having the possibility to use different models
and input for the analysis. The results can be inspected using the graphical
tool called There that has been developed alongside Where (Figure~\ref{fig:there}).


{\large\bfseries What?}\\

Where works on the basis of a pipeline. First observation data are read and
edited. Next, theoretical delays are calculated, before station positions are
estimated on the basis of those delays. Finally, results are written to disk in
proper formats (Figure~\ref{fig:architecture}).

The implementation of the individual models follow the 2010 IERS
Conventions~\cite{iers2010}, and when possible we have used software libraries
made available at the IERS web page~\cite{iers_software}. Table~\ref{tbl:models}
gives an overview of the models available for Where.

The estimation is done using a Kalman filter~\cite{mysen2017}. Currently we are
using continuous piecewise linear functions to model the clock and troposphere.
Different methods for detecting outliers are being explored~\cite{mysen2017b}.


{\large\bfseries Where?}\\

Currently VLBI is fully implemented (using NGS-files) and we are testing models
(see the box on the right) and the estimation. For SLR we have implemented all
models for orbit determination, but these still need improvements. For GNSS we
will first do a PPP-solution for GPS based on precise orbits. This is well under
way, although some models are still yet to be implemented. The DORIS analysis
has not yet been started.


{\large\bfseries When?}\\

We are an associated analysis center within the IVS and a report of our
activities the last two years will soon be published~\cite{kirkvik2017}. For
VLBI our short-term plans are to implement the VGOS-DB-format, better support
for TRFs for apriori positions (currently Where can only use the ITRF2014
without seismic deformations), and start testing of a production line.

The SLR development has more or less been on hold the last year due to maternity
leave, but will be taken up again this fall. The first priority will be to
improve the orbit integrator, and work on determining precise orbits.

For GNSS we will next finish implementing all necessary models and add a framework
for editing the data. DORIS development will be started as soon as resources allow.

\endinput
