In 2015 and 2016 Grzegorz Klopotek carried out a VLBI Analysis Software
Comparison Campaign (VASCC). The goal of the campaign was to compare computed
theoretical delays from different software packages. In total 11 different
software packages contributed to the campaign and the results where presented at
the IVS General Meeting in South Africa in 2016~\cite{klopotek2016}.

The Norwegian Mapping Authority provided solutions to VASCC using the legacy
software GEOSAT~\cite{kierulf2010}. However, as development of the new software
progressed we redid the analysis from the VASCC campaign using Where and
compared the results with the computed theoretical delays from the
c5++-software~\cite{hobiger2010}. The results from this comparison are presented
here.

The VASCC dataset includes two networks of stations: one with four stations on
the southern hemisphere and one with five stations on the northern hemisphere.
Virtual observations of one radio source for each network are scheduled every
minute for 15 days, from June 22nd to July 7th 2015. A leap second is introduced
at midnight June 30th.

The VASCC delay model includes geometric and gravitational delay from the IERS
2010 Conventions~\cite{iers2010}. Also the delay through the
troposphere~\cite{iers2010} (hydrostatic delay with GMF mapping function) and
delay due to thermal deformations and axis offset as described
by~\cite{nothnagel2009} are included. Delays due to cable calibration, the
ionosphere and clocks are ignored. Site displacement models includes solid
earth tides, solid earth pole tides, ocean tidal loading and ocean loading pole
tides according to~\cite{iers2010}. Atmospheric pressure loading and
eccentricity vectors are ignored. The apriori EOP time series is corrected for
ocean tides and liberation effects with periods less than two days according
to~\cite{iers2010}.

In the original campaign six software packages got sub-millimeter agreement when
comparing the RMS of the difference in theoretical delays. The absolute value of
the largest difference in residual was 2.68~mm and the smallest difference was
0.83~mm. One of the these six software packages was c5++~\cite{hobiger2010}.
Comparing c5++ with Where yields a RMS of 0.72~mm for the southern hemisphere
with a maximum residual of 2.26~mm. See Figure~\ref{fig:c5_where_sh}. Comparing
the northern hemisphere we get RMS of 0.68~mm and a maximum residual of
1.97~mm. Figure~\ref{fig:c5_where_nh} shows the difference in theoretical delays
between c5++ and Where for the Northern network.

These results indicate that the VLBI delay model in Where is consistent with
existing software packages and current conventions. When creating the new
software, the VASCC dataset has been valuable and an extension of the campaign
to include for instance the VMF1 mapping function and partial derivatives is
recommended. Other techniques such as SLR and DORIS might also benefit from
doing a similar comparison campaign.

\endinput
